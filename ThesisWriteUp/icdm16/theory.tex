\subsection{Learning-Theoretic Advantages}
\label{sec:learningtheory}

To formalize the intuitive advantage described in
Section~\ref{sec:intuition}, we present some simple theoretical
results that immediately follow from the literature.  This
section's purpose is not to advance the state of the art in 
learning theory, but to highlight the advantages that over-labeling
can provide.
We present observations in both the {\bf probably approximately correct
(PAC)}~\cite{v-tl-84} and {\bf exact}~\cite{a-lrsqc-87} models of learning. 
The results below focus on the 
% simple, but important 
case of learning concepts
that are unions of axis-parallel boxes 
% \textcolor{red}{(as in, e.g., decision trees \cite{})}
over real and  ordinal feature spaces.  
% Extending the results to other settings is an item of future work.

\subsubsection{PAC Learning with Active Over-labeling}
%\noindent {\bf 2.2.1 PAC Learning with Active Over-labeling}
In PAC, a learner is given parameters $0<\epsilon,\delta < 1/2$ 
and access to labeled training instances
drawn iid according to arbitrary distribution $\cal D$.  The
learner then outputs a hypothesis in polynomial time that,
with probability at least $1-\delta$, has error at most
$\epsilon$ on new instances drawn according to $\cal D$.  

\paragraph{Computational Complexity}
%\noindent {\bf 2.2.1.1 Computational Complexity}
% We begin by showing that in the passive learning setting, over-labeling
% offers advantages over standard labeling.
In the context  of computational complexity, we consider the case
% (in both the PAC and exact learning models)
of {\bf proper learning}\footnote{Note that the negative results described
below for both exact and PAC learning are only for proper learning.  One can
get positive results for these cases by allowing a logarithmic increase in the
number of boxes used, by applying the set cover approximation algorithm.},
in which the training instances are labeled by a
concept from $\cal C$ and the hypothesis inferred by the learner
is required to also be from $\cal C$.
%We consider the real space ${\cal X} = \mathbb{R}^d$.
%
%
We consider learning concepts  that are unions of
$k$ axis-parallel boxes in $\mathbb{R}^d$.  This
task is not properly PAC-learnable (i.e., learning $\cal C$ using $\cal C$)
if RP $\ne$ NP.
% a fact that can be obtained from results found in previous work.
\begin{obs}
The class of $k$-unions of axis-parallel boxes in $\mathbb{R}^d$ is not
properly PAC-learnable unless RP $=$ NP.
\end{obs}
{\bf Proof Sketch}: From Theorem 3.1.1 of Blumer et al.~\cite{behw-lvd-89}, 
concept class $\cal C$ is properly PAC learnable  iff there exists a 
randomized polynomial-time algorithm to find a hypothesis from $\cal C$
consistent with a sufficiently large labeled training sample 
$\cal X$ (called the {\bf consistent hypothesis problem}).
It is known to be NP-hard~\cite{behw-lvd-89,m-sncscp-xx} to find a smallest
set of rectangles to cover a set of points in $\mathbb{R}^d$ even
for $d=2$.  Thus, the consistent hypothesis problem for
$k$-unions of boxes is NP-hard, implying that one 
cannot properly PAC learn $k$-unions of boxes.  \hfill $\square$

In contrast, consider an over-labeling version of this learning problem,
% the $k$-unions of boxes problem,
in which each of the $k$ boxes
is a separate subconcept, as in Figure~\ref{fig:unionex}. 
Thus, examples from the $i$th box ($i=1,\ldots , k$) have a
fine-grained label (call it $I_i$) and all other 
examples are labeled `$-$'.
\begin{obs}
In the over-labeling setting, $k$-unions of boxes from $\mathbb{R}^d$
is properly PAC-learnable.
\end{obs}
{\bf Proof Sketch}: Let $m$ be the number of labeled training instances labeled
by the target concept of some $k$-union of boxes.
For each sub-concept, a consistent sub-hypothesis
(single bounding box)
can be learned from the fine-grained labels in time $O(dm)$.
The learner can learn each of the $k$ sub-concepts separately and
output their union in time $O(kdm)$, and this  union
will be consistent with all the labeled examples.
Thus, the consistent hypothesis problem can be solved in time polynomial in
$d$, $k$, and $m$.
Blumer et al.~\cite{behw-lvd-89} show that, if $m$ is sufficiently large,
then a consistent hypothesis $h$ will meet the PAC criteria.
Specifically, Equation~\ref{eqn:behw} below gives sufficient conditions on
$m$ for error bound $\epsilon$ and bound $\delta$ on probability of failure.
Because the over-labeling approach is finding a separate consistent hypothesis
for each of the $k$ fine-grained labels,
we apply Equation~\ref{eqn:behw}, but reduce
$\epsilon$ and $\delta$ each by a factor of $k$ to account for this.
This yields a polynomial bound on $m$.
% that is polynomial in all relevant parameters.
 \hfill $\square$

% \subsubsection{Sample Complexity}
% In the context of sample complexity, we look at passive PAC learning of unions of $k$
% axis-parallel boxes in $\mathbb{R}^d$.  The
% classic approach~\cite{behw-lvd-89} to PAC learning a function class $\cal C$ is to
% draw a labeled sample $\cal X$ of size  at least
% \begin{equation}
% m(\epsilon,\delta) = \max \left( {2 \over \epsilon} \log {2 \over \delta}, {8D \over \epsilon} \log {13
% \over \epsilon}  \right)
% \enspace ,
% \label{eqn:behw}
% \end{equation}
% where $\epsilon$ and $\delta$ are the PAC parameters and $D$ is
% the VC-dimension (VCD) of $\cal C$,
% and to then find some function $c \in {\cal C}$ consistent with
% $\cal X$.  
% 
% The VC-dimension of the class of single $d$-dimensional axis-aligned boxes is $2d$.  Based on
% Blumer et al.~\cite{behw-lvd-89}, then, the VCD of the $k$-union of
% $d$-dimensional axis-aligned boxes is
% $\Theta(d  k \log k)$.  Thus, to directly PAC-learn $k$-unions of $d$-dimensional boxes,
% it suffices to find a hypothesis consistent with 
% \[
% m_{coarse}(\epsilon,\delta) = \Theta \left( {1 \over \epsilon} \log {1 \over \delta}
% + {d k \log k \over \epsilon} \log {1 \over \epsilon}  \right)
% \enspace 
% \]
% labeled examples\footnote{Note that we are not considering the time complexity of finding such
% a consistent hypothesis, only the number of training instances.}.
% In contrast, learning the class of individual $d$-dimensional boxes reduces
% the VCD from $\Theta(d  k \log k)$ to $\Theta(d)$.  However, since the learning
% process is repeated $k$ times, one needs to learn each individual box with a smaller value of
% $\epsilon$ and a smaller value of $\delta$,
% to allow for the union bound to be applied across all $k$
% fine-grained hypotheses.  This yields a sample complexity of 
% \[
% m_{fine}(\epsilon/k,\delta/k) = \Theta \left( {k \over \epsilon} \log {k \over \delta}
% + {d k  \over \epsilon} \log {k \over \epsilon}  \right)
% \enspace 
% ,
% \]
% which grows more slowly than $m_{coarse}$.
% %  since  $m_{coarse}-m_{fine}$ is positive.

\paragraph{Label Complexity}
%\noindent {\bf 2.2.1.2 Label Complexity}
We now consider label complexity, in which one wants to minimize the 
number of labels purchased by a pool-based active learning algorithm.
We work in a model where we are given a size-$m$ set of training data $U$,
but initially the labels are missing. When seeking a PAC algorithm for learning, one
can apply a standard result from Blumer et al.~\cite{behw-lvd-89} that says if
the algorithm efficiently finds a hypothesis from $\cal C$ that is
consistent with $U$, which is
drawn iid from fixed distribution $\cal D$ and is of size at least
\begin{equation}
m(\epsilon,\delta) = \max \left( {2 \over \epsilon} \log {2 \over \delta}, {8D \over \epsilon} \log {13
\over \epsilon}  \right)
\label{eqn:behw}
\end{equation}
(where $D$ is the {\bf VC dimension} of ${\cal C}$),
then with probability $ \ge 1-\delta$, the hypothesis will have error
at most $\epsilon$.  If the instances of $U$ are 
unlabeled, the goal in active learning is to purchase as few as possible
labels of
instances of $U$ and still guarantee a hypothesis consistent with
all of $U$ (including the yet unlabeled ones), which would yield a PAC result. 

For this example, we focus on what we term the {\bf disjoint $k$-intervals
problem}.
% of learning axis-parallel boxes on the real line.
I.e., $\cal C$ is the set of unions of 
$ \le k$ disjoint intervals on $\mathbb{R}$. When a coarse-grained label
of instance $x \in U$ is purchased, it returns
`$+$' if $x$ lies in one of the $k$ target intervals and `$-$' otherwise.
When a fine-grained label of $x$ is purchased, the label is an indicator
of which of the $k$ target intervals it lies in ($I_1,\ldots,I_k$) or `$-$' if
it does not lie in any interval.  We assume that there is
at least one point from $U$ in each interval $I_j$ and that there is
at least  one point from $U$ between each adjacent pair of intervals.
% (otherwise, those empty intervals/gaps between intervals are irrelevant in the PAC sense).

In the following two observations, we 
bound the number of purchases needed in each labeling scheme to find a consistent
hypothesis.  Since the total number of instances needed for PAC learning
(per Equation~\ref{eqn:behw}) differs between them (due to different VC
dimensions), in the next two observations, we use $m_c$ for the 
number of instances in coarse-grained learning and $m_f$ needed
for fine-grained.

Assume that, for each target interval, there is
one instance of  $U$ that is pre-labeled for free.  I.e., in the 
coarse-grained case, there are $k$ instances labeled `$+$' (one in each target
interval) and in the fine-grained case there is one instance labeled $I_1$,
one labeled $I_2$, etc. 

\begin{obs}
The consistent hypothesis problem on disjoint $k$-intervals
with coarse-grained labels on $m_c$ instances requires
$\Omega(m_c)$ label purchases in the worst case.
\end{obs}
{\bf Proof Sketch}:  
The algorithm must find the
left and right boundaries of each of the $k$ target intervals, which
is tantamount to identifying the leftmost and rightmost negatively
labeled points between each consecutive pair of intervals.  Consider
two consecutive intervals $I_j$ and $I_\ell$. 
% that, when taken together and their intervening gap, contain the largest number of
% points from $U$ of all consecutive pairs of intervals.  Since
% it is the largest number of points, this count has to be $\Omega(m/k)$
% but could be $O(m)$ in the worst case.  
In searching for
the negative points from $U$ between $I_j$ and
$I_\ell$, the learner must purchase the label of some point between $x_j$ and
$x_\ell$, where $x_j$ and $x_\ell$ are the pre-labeled points from
$U$ from $I_j$ and $I_\ell$, respectively.
In the worst case, every query will result in a 
response of `$+$', until only one remains to be labeled
`$-$'.  Summed over all pairs of intervals, this 
requires $\Omega(m_c)$ purchases in the worst case. 
 \hfill $\square$

\begin{obs}
The consistent hypothesis problem on disjoint $k$-intervals
with fine-grained labels on $m_f$ instances
requires $O(k \log m_f)$ queries in the worst case.
\end{obs}
{\bf Proof Sketch}:  
An algorithm in the active over-labeling setting 
can perform a binary search between $x_j$ and $x_\ell$  (labeled
$I_j$ and $I_\ell$ rather than simply `$+$') until a negatively labeled
instance $x_-$ is found.  When
that is done, the learner can simply perform two binary searches: one between $x_-$ and
the right-most point in $I_j$ and one between $x_-$ and
the left-most point in $I_\ell$. This requires at most $O(\log m_f)$ queries
per pair of adjacent intervals, for a total of $O(k \log m_f)$ queries. 
 \hfill $\square$

% Thus, we see that active over-labeling requires exponentially fewer queries
% (in $m$) in the worst case when compared with standard active learning.

% classic approach~\cite{behw-lvd-89} to PAC learning a function class $\cal C$ is to
% draw a labeled sample $\cal X$ of size  at least
% \begin{equation}
% m(\epsilon,\delta) = \max \left( {2 \over \epsilon} \log {2 \over \delta}, {8D \over \epsilon} \log {13
% \over \epsilon}  \right)
% \enspace ,
% \label{eqn:behw}
% \end{equation}
% where $\epsilon$ and $\delta$ are the PAC parameters and $D$ is
% the VC-dimension (VCD) of $\cal C$,
% and to then find some function $c \in {\cal C}$ consistent with
% $\cal X$.  

To bound $m_c$, we use Equation~\ref{eqn:behw} with
VC dimension~\cite{behw-lvd-89} $D=2k$
and get (ignoring
the typically smaller first term) a number of purchases
$\Omega(m_c)= \Omega ((k / \epsilon ) (\log 1/\epsilon))$.
To bound $m_f$, note that we have $k$ independent learning problems, each a
single box.  Thus, we can use VC dimension~\cite{behw-lvd-89} $D=2$,
but the parameters $\epsilon$ and $\delta$
must each be reduced by a factor of $k$, since the errors of these hypotheses
accumulate.  Further, we must apply the learning process $k$ times, so (again
ignoring the first term) $m_f=O((k^2/\epsilon) \log (k/\epsilon))$, so 
our worst-case upper bound of purchases is
$O(k \log (k/\epsilon) + k \log \log (k/\epsilon))$. Both bounds grow linearly in $k$ but the
coarse-grained learner's bound is worse by a factor  exponential in $1/\epsilon$.

Simply put, the advantage that the fine-grained approach has comes from
the fact that, for positively-labeled instance $x \in U$, the fine-grained
label indicates the interval that $x$ lies in, while in the coarse-grained
approach, the label is simply `$+$'.  The distinct fine-grained label 
given by each interval allows for a binary search for interval boundaries,
hence the logarithmic dependence on $m_f$. In contrast,
the homogeneous `$+$' label across all intervals
for the coarse-grained labels can force a number of purchases linear in $m_c$.

% Finally, we can show that the worst-case number of label purchases required by a standard active learner
% is bounded above that of an active over-labeling learner.
% Applying Equation~\ref{eqn:behw},
% since the VC dimension $D$ of the union of $k$ intervals is $2k$, we will need
% $U$ of size 
% \[
% m \ge \max \left( {2 \over \epsilon} \log {2 \over \delta}, {16 k \over \epsilon} \log {13
% \over \epsilon}  \right)
% \enspace .
% \]
% The second term of the max is the dominant one (unless $\delta$ is exponentially small
% in $k$), so we focus on it.  This gives a worst-case lower bound 
% of purchases for a coarse-grained learner to be
% $\Omega ((k / \epsilon ) (\log 1/\epsilon))$,
% while our worst-case upper bound of purchases for a fine-grained learner is
% $O(k \log (k/\epsilon) + k \log \log (1/\epsilon))$. Both are linear in $k$ but the
% coarse-grained learner's bound is worse by a factor  exponential in $1/\epsilon$.

\subsubsection{Exact Learning with Active Over-Labeling}
%\noindent {\bf 2.2.2 Exact Learning with Active Over-Labeling}
We now illustrate the computational complexity advantages of
active over-labeling in the exact learning setting.  In 
exact learning, the learner gets access to two oracles: a
{\bf membership query} (MQ) oracle and an {\bf equivalence query}
(EQ) oracle.  An efficient learner will learn the exact identity
of the target concept in time and number of queries that are
polynomial in the problem size.  When the learner poses an EQ, it
passes to the oracle a hypothesis $h \in {\cal C}$ that it thinks is exactly
equivalent to the target concept, i.e., that will label all instances
correctly.  The oracle either responds that the hypothesis
is exactly correct or gives to the learner a counterexample, which
is an instance on which $h$ is wrong.  An MQ oracle receives from
the learner an instance $x$ and provides $x$'s label.  This is similar
to a pool-based active learning model, except that in the MQ model, the
instances can be arbitrary, while in pool-based active learning,
the instances must come from a pre-specified set.

We consider
%the passive (where all instances are labeled)
proper learning of $k$-unions of disjoint axis-parallel boxes,
in a bounded, discretized, $d$-dimensional instance space $\{0,\ldots,t-1\}^d$.
% (e.g., Figure~\ref{fig:unionex} using integer coordinates).

\begin{obs}
With over-labeling, disjoint $k$-unions of boxes 
% in a discrete space
can be exactly learned with
$O(k)$ EQs and $O(kd \log t)$ MQs and time polynomial in the number of queries.
\end{obs}
{\bf Proof Sketch}:
Using fine-grained labels for $k$ distinct fine-grained hypotheses (each using one 
box), one can exactly learn each box $j$ individually with one EQ (to get an
instance in box $j$)
and $O(d \log t)$ MQs (for binary search to find the box $j$'s $2d$
boundaries), for a
total of $O(k)$ EQs and $O(kd \log t)$ MQs and polynomial time.
%  polynomial in the number of queries.
 \hfill $\square$

This contrasts with a result from Bshouty and Burroughs~\cite{bb-plapc-03}
that one cannot exactly properly learn $k$-unions of axis-parallel boxes
when (constant) $d>2$ unless P $=$ NP.  I.e., while one 
can learn $k$-unions with $O(d \log k)$-unions, one cannot 
efficiently learn $k$-unions with $k$-unions  if P $\ne$ NP.  
%
Note that our positive result for over-labeling works for non-constant $d$, while the
hardness result for direct proper learning holds even for constant $d$. 
%Thus, there is clearly an advantage to having more refined label information in the form of which of the $k$ boxes an instance lies in.

% The above results show that in both the exact and PAC learning setting, and in active and passive learning,
% the over-labeling approach offers theoretical advantages over standard learning.  We now present
% our general method for active over-labeling, and evaluate it experimentally.
